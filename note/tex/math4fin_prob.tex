\section{Probability}

\subsection{Set theory}
Under moving...



\subsection{Permutation and Combination}
Under moving...



\subsection{Probability}
\subsubsection{Basic}
Under moving...


\subsubsection{Conditional probability and Bayes’s formula}
Under moving...



\subsection{Random variables}
\subsubsection{Basis}
Under moving...


\subsubsection{Properties}
Under moving...



\subsection{Discrete random variables}
\subsubsection{Basis}
Under moving...


\subsubsection{Uniform discrete random variable}
Under moving...


\subsubsection{Binomial random variable}
Under moving...


\subsubsection{Geometric random variable}
Under moving...


\subsubsection{Negative binomial random variable}
Under moving...


\subsubsection{Hyper-geometric random variable}
Under moving...


\subsubsection{Poisson random variable}
Under moving...



\subsection{Continuous random variables}
\subsubsection{Basis}
Under moving...


\subsubsection{Expected value}
If $X$ is a continuous random variable having a pdf $f(x)$ then the \textbf{expected value}\index{expected value} of $X$ is given by: 
\begin{equation}
    E[X] = \int_{-\infty}^\infty x f(x) dx
\end{equation}


\subsubsection{Cumulative distribution function (cdf)}
The \textbf{cumulative distribution function}\index{cumulative distribution function}\index{cdf}, $F$, of the random variable $X$ is defined for all real numbers $b$, by:
\begin{equation}
    F(b) = \mathbf{P}\{X \leq b\}
\end{equation}


\subsubsection{Probability density function (pdf)}
We say $X$ admits a \textbf{probability density function}\index{probability density function}\index{pdf} or \textbf{density}\index{density} if:
\begin{equation}
   \mathbf{P}\{X \leq b\} = F(b) = \int_{-\infty}^b f(x) dx 
\end{equation}
for some non-negative function $f$.


\subsubsection{Continuous uniform distribution function}
Under moving...


\subsubsection{Normal random variable}
$X$ is a \textbf{normal random variable}\index{normal random variable} with parameter $\mu$ and $\sigma^2$ if the density of $X$ is given by:
\begin{equation}
    f(x) = \frac{1}{\sqrt{2\pi}\sigma} e^\frac{-(x-\mu)^2}{2\sigma^2} \quad -\infty < x < \infty
\end{equation}
Thus, the cdf of a \textbf{standard random variable}\index{standard random variable}, i.e. one with mean $\mu = 0$ and variance $\sigma^2 = 1$, is given by:
\begin{equation}
    N(x) = \frac{1}{\sqrt{2\pi}} \int_{-\infty}^{0} e^\frac{-y^2}{2} dy
\end{equation}

The random variable $X$ is \textit{log-normally distributed} if for some normally distributed variable $Y$, $X=e^Y$, that is $\ln(X)$ is normally distributed. 


\subsubsection{Exponential random variable}
Under moving...


\subsubsection{Gamma distribution}
Under moving...



\subsection{Relationship between random variables}
\subsubsection{Basis}
Under moving...


\subsubsection{Independent random variables}
Under moving...


\subsubsection{Conditional distribution}
Under moving...


\subsubsection{Joint pdf of functions}
Under moving...


\subsubsection{Expected value of a function}
Under moving...


\subsubsection{Covariance}
Under moving...


\subsubsection{Coefficient of correlation}
Under moving...



\subsection{Miscellaneous}
\subsubsection{Central tendency}
Under moving...


\subsubsection{Moment generating functions}
Under moving...


\subsubsection{Central limit theorem}
Under moving...


\subsubsection{Special series}
Under moving...

